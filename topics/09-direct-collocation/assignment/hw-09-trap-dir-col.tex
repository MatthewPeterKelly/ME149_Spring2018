% This file contains the header data for all assignment files
\documentclass[onecolumn, 11pt]{article}

\usepackage{bbold}   %Get fancy double struck math notation for sets
\usepackage{cite}
\usepackage{cleveref}
\usepackage{color}
\usepackage{courier}   %Have code written out nicely
\usepackage{float}
\usepackage[top=1in, bottom=1in, left=1in, right=1in]{geometry}
\usepackage{graphicx} % handles graphics and figures
\usepackage{hyperref}
\usepackage{listings}
\usepackage{mathtools,bm}
\usepackage{multicol}

\catcode`\^^M=10      %  Makes blank lines meaningless, force use of \par

\definecolor{magenta}{rgb}{0.8,0.0,1.0}
\definecolor{darkGreen}{rgb}{0.0,0.4,0.0}
\definecolor{blue}{rgb}{0.0, 0.0, 0.9}
\definecolor{purple}{rgb}{0.7, 0.0, 0.7}
\definecolor{darkGreen}{rgb}{0.0,0.4,0.0}

\newcommand{\quotes}[1]{``#1''}
\newcommand{\todo}[1]{{\color{magenta}\par {[{\bf ToDo: } {\em #1}} ] \\    }}

\newcommand{\norm}[1]{\left\lVert#1\right\rVert}

%%%%%%%%%%%%%%%%%%%%%%%%%%%%%%%%%%%%%%%%%%%%%%%%%%%%%%%%%%%%%%%%%%%%%%%%%%%%%%%
% NOTE:
%
% The following block of commands is used to format Matlab code blocks for the
% listings package. This block of code is based on two examples:
%  -->   https://gist.github.com/eyliu/120689
%  -->   http://links.tedpavlic.com/ascii/homework_new_tex.ascii
%
% Import file block using:
%   \lstinputlisting{fileName.m}
%
\lstloadlanguages{Matlab}
%
\lstset{language=Matlab,                        % Use MATLAB
        frame=single,                           % Single frame around code
        basicstyle=\small\ttfamily,             % Use small true type font
        keywordstyle=[1]\color{blue}\bfseries,  % MATLAB functions bold and blue
        keywordstyle=[2]\color{purple},         % MATLAB function arguments purple
        keywordstyle=[3]\color{blue}\underbar,  % User functions underlined and blue
        identifierstyle=,                       % Nothing special about identifiers
                                                % Comments small dark green courier
        commentstyle=\usefont{T1}{pcr}{m}{sl}\color{darkGreen}\small,
        stringstyle=\color{darkGreen},            % Strings are purple
        showstringspaces=false,                 % Don't put marks in string spaces
        tabsize=4,                              % 4 spaces per tab
        %
        %%% Put standard MATLAB functions not included in the default
        %%% language here
        morekeywords={xlim,ylim,var,alpha,factorial,poissrnd,normpdf,normcdf},
        %
        %%% Put MATLAB function parameters here
        morekeywords=[2]{on, off, interp},
        %
        %%% Put user defined functions here
        morekeywords=[3]{FindESS, homework_example},
        %
        morecomment=[l][\color{blue}]{...},     % Line continuation (...) like blue comment
        numbers=left,                           % Line numbers on left
        firstnumber=1,                          % Line numbers start with line 1
        numberstyle=\tiny\color{blue},          % Line numbers are blue
        stepnumber=5                            % Line numbers go in steps of 5
        }
%
%%%%%%%%%%%%%%%%%%%%%%%%%%%%%%%%%%%%%%%%%%%%%%%%%%%%%%%%%%%%%%%%%%%%%%%%%%%%%%%


%========================================================================
\title{Assignment 09:  Trapezoid Direct Collocation}
\date{Assigned:  March 27  ---  Due:  April 4 at 11:55pm}
\author{Optimal Control for Robotics}
%========================================================================
\begin{document}
\maketitle
%=================================================

\section*{Introduction}

In this assignment you will use the trapezoid method for direct collocation to
compute the minimal-torque swing-up for a simple pendulum,
and the minimal-thrust flip maneuver for a planar quadrotor model.

\section*{Implementation Details}
Most of the code to implement these two optimizations is included in the
assignment. You only need to implement two sub-functions within the file
\texttt{dirColBvpTrap.m}.
The first computes the objective function value and the second evaluates the
nonlinear constraints.
The \texttt{dirColBvpTrap()} function calls a few functions from the code library,
so make sure that you've added that to your Matlab path before starting this assignment.

\par
Your code must be written well, both for computational efficiency and readability.

\vspace{-0.0em} \begin{itemize}  \setlength\itemsep{0em} \setlength\itemindent{18pt}

  \item Your code should be fully vectorized: no for loops are allowed.

  \item Use the function \texttt{unpackDecVars} to obtain the
  state and control at the grid points.
  This makes the code more readable and avoids bugs relating to indexing into decVars.

  \item Make use of the data provided in the problem description struct.
  Many useful quantities are pre-computed and saved into this data structure.

\end{itemize}

\section*{Coding Challenge}

If you want a challenge, then start this assignment by deleting some or all
parts of the template code, leaving only the function declaration and
documentation (first 75 lines of code). If you choose this option, then please
add a note to the function documentation, along with a description of how you
organized your code.

Your code should still follow the good coding practices described above,
although you do not need to implement the specific sub-functions that are
in the template.

\section*{Deliverables}

Implement the function \texttt{dirColBvpTrap.m} using the template provided.

%=================================================
\end{document}
