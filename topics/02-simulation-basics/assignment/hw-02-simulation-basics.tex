% This file contains the header data for all assignment files
\documentclass[onecolumn, 11pt]{article}

\usepackage{bbold}   %Get fancy double struck math notation for sets
\usepackage{cite}
\usepackage{cleveref}
\usepackage{color}
\usepackage{courier}   %Have code written out nicely
\usepackage{float}
\usepackage[top=1in, bottom=1in, left=1in, right=1in]{geometry}
\usepackage{graphicx} % handles graphics and figures
\usepackage{hyperref}
\usepackage{listings}
\usepackage{mathtools,bm}
\usepackage{multicol}

\catcode`\^^M=10      %  Makes blank lines meaningless, force use of \par

\definecolor{magenta}{rgb}{0.8,0.0,1.0}
\definecolor{darkGreen}{rgb}{0.0,0.4,0.0}
\definecolor{blue}{rgb}{0.0, 0.0, 0.9}
\definecolor{purple}{rgb}{0.7, 0.0, 0.7}
\definecolor{darkGreen}{rgb}{0.0,0.4,0.0}

\newcommand{\quotes}[1]{``#1''}
\newcommand{\todo}[1]{{\color{magenta}\par {[{\bf ToDo: } {\em #1}} ] \\    }}

\newcommand{\norm}[1]{\left\lVert#1\right\rVert}

%%%%%%%%%%%%%%%%%%%%%%%%%%%%%%%%%%%%%%%%%%%%%%%%%%%%%%%%%%%%%%%%%%%%%%%%%%%%%%%
% NOTE:
%
% The following block of commands is used to format Matlab code blocks for the
% listings package. This block of code is based on two examples:
%  -->   https://gist.github.com/eyliu/120689
%  -->   http://links.tedpavlic.com/ascii/homework_new_tex.ascii
%
% Import file block using:
%   \lstinputlisting{fileName.m}
%
\lstloadlanguages{Matlab}
%
\lstset{language=Matlab,                        % Use MATLAB
        frame=single,                           % Single frame around code
        basicstyle=\small\ttfamily,             % Use small true type font
        keywordstyle=[1]\color{blue}\bfseries,  % MATLAB functions bold and blue
        keywordstyle=[2]\color{purple},         % MATLAB function arguments purple
        keywordstyle=[3]\color{blue}\underbar,  % User functions underlined and blue
        identifierstyle=,                       % Nothing special about identifiers
                                                % Comments small dark green courier
        commentstyle=\usefont{T1}{pcr}{m}{sl}\color{darkGreen}\small,
        stringstyle=\color{darkGreen},            % Strings are purple
        showstringspaces=false,                 % Don't put marks in string spaces
        tabsize=4,                              % 4 spaces per tab
        %
        %%% Put standard MATLAB functions not included in the default
        %%% language here
        morekeywords={xlim,ylim,var,alpha,factorial,poissrnd,normpdf,normcdf},
        %
        %%% Put MATLAB function parameters here
        morekeywords=[2]{on, off, interp},
        %
        %%% Put user defined functions here
        morekeywords=[3]{FindESS, homework_example},
        %
        morecomment=[l][\color{blue}]{...},     % Line continuation (...) like blue comment
        numbers=left,                           % Line numbers on left
        firstnumber=1,                          % Line numbers start with line 1
        numberstyle=\tiny\color{blue},          % Line numbers are blue
        stepnumber=5                            % Line numbers go in steps of 5
        }
%
%%%%%%%%%%%%%%%%%%%%%%%%%%%%%%%%%%%%%%%%%%%%%%%%%%%%%%%%%%%%%%%%%%%%%%%%%%%%%%%


%========================================================================
\title{Assignment 2:  Simulation Basics}
\date{Assigned:  Jan. 23  ---  Due:  Feb. 1}
\author{Optimal Control For Robotics}
%========================================================================
\begin{document}
\maketitle
%=================================================

\section*{Introduction}\

In this assignment you will implement Euler's method and
use it to simulate two simple dynamical systems.
The first system is an ideal passive pendulum,
one of the simplest non-linear systems that corresponds to a physical mechanism.
The second system is the Lorenz system,
which is highly non-linear and generates the famous Lorenz Attractor.

\par This assignment includes a set of template files to guide you in writing the
simulation function, dynamics functions, and top-level code to generate the plots.
\vspace{-0.6em} \begin{itemize}  \setlength\itemsep{0em}
  \item \texttt{EulerMethodSimulate.m} --- simulate a dynamical system using Euler's method
  \item \texttt{PendulumDynamics.m} --- compute the system dynamics for a simple pendulum
  \item \texttt{LorenzDynamics.m} --- compute the system dynamics for the Lorenz system
  \item \texttt{prob\_03\_studentName.m} --- run the simulation of the pendulum and make plots
  \item \texttt{prob\_04\_studentName.m} --- run the simulation of the Lorenz system and make plots
\end{itemize}

You will need to complete each of these files by replacing the \texttt{\%\%\%\% TODO} markings with
correct implementations in each case. There are detailed comments throughout the template code
to indicate what each section of the code should do. The template files are described in
detail at the end of this assignment, and included as stand-alone Matlab files as well.

\section*{Write-Up}

You will submit a single (short) write-up for this assignment:
\vspace{-0.0em} \begin{itemize}  \setlength\itemsep{0em} \setlength\itemindent{18pt}
  \item Header: full name, studentName, date, assignment name and number
  \item List any other students that you worked with.
  \item How long did this assignment take you?
\end{itemize}

\section*{Deliverables}
\begin{itemize}
  \item Fully implement all five template files and submit them.
  \item Submit the plots that you generated for each simulation: \\
        \texttt{prob\_03\_studentName.pdf}   and   \texttt{prob\_04\_studentName.pdf}
  \item Submit the write-up for the assignment:   \texttt{hw\_02\_studentName.txt}
  \item All code should be clearly written and well documented. Figures should include labels and titles.
  \item All files should besubmitted through Tufts Trunk (as a single compressed file)
\end{itemize}


%~~~~~~~~~~~~~~~~~~~~~~~~~~~~~~~~~~~~~~~~~~~~~~~~~~~~~~~~~~~~~~~~~~~~~~~~~%
\pagebreak
\section*{\texttt{EulerMethodSimulation.m}}

The key line that you will write in this file will implement Euler's Method for integration,
given below, where $\bm{z}_k$ is the state at the current time step, $\bm{z}_{k+1}$ is the state
at the next time step, $h$ is the duration of the time step, and $\dot{\bm{z}}_k$ is the time-derivative
of the state at the current time step.
Notice that this function is general-purpose: it can be used to simulate any dynamical system.

\begin{equation}
  \bm{z}_{k+1} = \bm{z}_k + h \dot{\bm{z}}_k
\end{equation}

\lstinputlisting{EulerMethodSimulation.m}

%~~~~~~~~~~~~~~~~~~~~~~~~~~~~~~~~~~~~~~~~~~~~~~~~~~~~~~~~~~~~~~~~~~~~~~~~~%
\pagebreak
\section*{\texttt{PendulumDynamics.m}}

The dynamics of a simple pendulum can be modeled as:

\begin{equation}
  \ddot{q} = -\sin(q)
\end{equation}

Euler's Method for simulation, like most simulation methods, requires the dynamical system to be
in first-order form. To to this, we will introduce a new variable $\omega = \dot{q}$ to represent
angular rate. This allows us to write a second-order equation as a system of two first-order equations:

\begin{align}
  &  \dot{q} = \omega \\
  &  \dot{\omega} = -\sin(q)
\end{align}

You will notice a tilde (\texttt{\~}) in the argument list for this function.
This tells Matlab that our function does not use this argument
and allows it the compiler to do some optimization to improve run-time speed.
We keep the argument here beause simulation functions
(such as \texttt{EulerMethodSimulate()} and \texttt{ode45()})
often require the first argument of the dynamics function to be time.

\lstinputlisting{PendulumDynamics.m}

%~~~~~~~~~~~~~~~~~~~~~~~~~~~~~~~~~~~~~~~~~~~~~~~~~~~~~~~~~~~~~~~~~~~~~~~~~%
\pagebreak
\section*{\texttt{LorenzDynamics.m}}

This function implements the equations for the Lorenz system. For a special combination of
parameters this system produces a famous fractal known as the Lorenz attractor.
The equations for this system are given below, where $x$, $y$, and $z$ are the state of the
system, and $\sigma=10$, $\rho=29$, and $\beta = 8/3$ are system parameters.

\begin{align}
  &  \dot{x} = \sigma (y-x) \\
  &  \dot{y} = x (\rho - z) - y \\
  &  \dot{z} = x y - \beta z
\end{align}

\lstinputlisting{LorenzDynamics.m}

%~~~~~~~~~~~~~~~~~~~~~~~~~~~~~~~~~~~~~~~~~~~~~~~~~~~~~~~~~~~~~~~~~~~~~~~~~%
\pagebreak
\section*{\texttt{prob\_03\_studentName.m}}

This function sets up and runs a simulation of the simple pendulum system and then generates a plot
of the resulting simulation. Let's assume that the pendulum is using SI units.
Follow the directions in the template code for simulation and plotting details.
When you submit this file you should replace \texttt{studentName} with your full name.

\lstinputlisting{prob_03_studentName.m}

%~~~~~~~~~~~~~~~~~~~~~~~~~~~~~~~~~~~~~~~~~~~~~~~~~~~~~~~~~~~~~~~~~~~~~~~~~%
\pagebreak
\section*{\texttt{prob\_04\_studentName.m}}

This function sets up and runs a simulation of the Lorenz System and then generates a plot of the
Lorenz attractor. We will use dimensionless units.
Follow the directions in the template code for simulation and plotting details.

\lstinputlisting{prob_04_studentName.m}

%~~~~~~~~~~~~~~~~~~~~~~~~~~~~~~~~~~~~~~~~~~~~~~~~~~~~~~~~~~~~~~~~~~~~~~~~~%
\pagebreak

%=================================================
\end{document}
