% This file contains the header data for all assignment files
\documentclass[onecolumn, 11pt]{article}

\usepackage{bbold}   %Get fancy double struck math notation for sets
\usepackage{cite}
\usepackage{cleveref}
\usepackage{color}
\usepackage{courier}   %Have code written out nicely
\usepackage{float}
\usepackage[top=1in, bottom=1in, left=1in, right=1in]{geometry}
\usepackage{graphicx} % handles graphics and figures
\usepackage{hyperref}
\usepackage{listings}
\usepackage{mathtools,bm}
\usepackage{multicol}

\catcode`\^^M=10      %  Makes blank lines meaningless, force use of \par

\definecolor{magenta}{rgb}{0.8,0.0,1.0}
\definecolor{darkGreen}{rgb}{0.0,0.4,0.0}
\definecolor{blue}{rgb}{0.0, 0.0, 0.9}
\definecolor{purple}{rgb}{0.7, 0.0, 0.7}
\definecolor{darkGreen}{rgb}{0.0,0.4,0.0}

\newcommand{\quotes}[1]{``#1''}
\newcommand{\todo}[1]{{\color{magenta}\par {[{\bf ToDo: } {\em #1}} ] \\    }}

\newcommand{\norm}[1]{\left\lVert#1\right\rVert}

%%%%%%%%%%%%%%%%%%%%%%%%%%%%%%%%%%%%%%%%%%%%%%%%%%%%%%%%%%%%%%%%%%%%%%%%%%%%%%%
% NOTE:
%
% The following block of commands is used to format Matlab code blocks for the
% listings package. This block of code is based on two examples:
%  -->   https://gist.github.com/eyliu/120689
%  -->   http://links.tedpavlic.com/ascii/homework_new_tex.ascii
%
% Import file block using:
%   \lstinputlisting{fileName.m}
%
\lstloadlanguages{Matlab}
%
\lstset{language=Matlab,                        % Use MATLAB
        frame=single,                           % Single frame around code
        basicstyle=\small\ttfamily,             % Use small true type font
        keywordstyle=[1]\color{blue}\bfseries,  % MATLAB functions bold and blue
        keywordstyle=[2]\color{purple},         % MATLAB function arguments purple
        keywordstyle=[3]\color{blue}\underbar,  % User functions underlined and blue
        identifierstyle=,                       % Nothing special about identifiers
                                                % Comments small dark green courier
        commentstyle=\usefont{T1}{pcr}{m}{sl}\color{darkGreen}\small,
        stringstyle=\color{darkGreen},            % Strings are purple
        showstringspaces=false,                 % Don't put marks in string spaces
        tabsize=4,                              % 4 spaces per tab
        %
        %%% Put standard MATLAB functions not included in the default
        %%% language here
        morekeywords={xlim,ylim,var,alpha,factorial,poissrnd,normpdf,normcdf},
        %
        %%% Put MATLAB function parameters here
        morekeywords=[2]{on, off, interp},
        %
        %%% Put user defined functions here
        morekeywords=[3]{FindESS, homework_example},
        %
        morecomment=[l][\color{blue}]{...},     % Line continuation (...) like blue comment
        numbers=left,                           % Line numbers on left
        firstnumber=1,                          % Line numbers start with line 1
        numberstyle=\tiny\color{blue},          % Line numbers are blue
        stepnumber=5                            % Line numbers go in steps of 5
        }
%
%%%%%%%%%%%%%%%%%%%%%%%%%%%%%%%%%%%%%%%%%%%%%%%%%%%%%%%%%%%%%%%%%%%%%%%%%%%%%%%


%========================================================================
\title{Assignment 3:  Simulation using Runge--Kutta}
\date{Assigned:  Jan. 30  ---  Due:  Feb. 8}
\author{Tufts ME 149:  Optimal Control For Robotics}
%========================================================================
\begin{document}
\maketitle
%=================================================

\section*{Introduction}

In this assignment you will implement a general-purpose simulation function
which implements a few different Runge--Kutta integration methods.
You will then use this function to generate simulations for two different dynamical systems:
a driven-damped pendulum and a passive double pendulum.
The final part of the assignment is to create two plots, one for each system,
showing the solution as well as the absolute error for each method.

\section*{Write-up}

Instead of submitting a separate write-up,
please fill-in the header block in \texttt{hw\_03\_studentName.m}.
Like previous assignments, this includes
your name, information about the assignment, collaborators and references,
how long the assignment took to complete, and
concludes with an outline of your code.

\section*{Deliverables}

Please upload each of the following files to Trunk ---
you do not need to zip them to a single file before uploading.
If you choose to create any additional Matlab files as part of you assignment,
then please upload those as well.
You do not need to include any files (or functions)
that are part of the \texttt{codeLibrary} for this course.

\vspace{-0.6em} \begin{itemize}  \setlength\itemsep{0em}
 \item \texttt{hw\_03\_studentName.m} --- main function for the assignment
 \item \texttt{runSimulation.m} --- general simulation function using Runge--Kutta methods
 \item \texttt{hw\_03\_driven\_damped\_pendulum.pdf} --- plot for the driven damped pendulum results
 \item \texttt{hw\_03\_passive\_double\_pendulum.pdf} --- plot for the passive double pendulum results
\end{itemize}

See the following pages for additional details on how to create each file.

%=================================================
\pagebreak

\section*{Details: \texttt{runSimulation.m}}

\lstinputlisting{runSimulation.m}

\subsection*{Requirements}

This function provides an interface that makes it easy to run a simulation
with a variety of different integration methods.
You need to implement Euler's method,
at least one of the second-order methods,
and ``The'' forth-order Runge--Kutta method.
While working on your implementation please follow good programming practices.

\subsection*{Programming Challenge}

The easiest way to implement this function is to have a separate sub-function
for each of the three method orders (first, second and forth).
You will receive full credit for this style of implementation.

\par
An alternate implementation is to write a truely general-purpose sub-function
that will work for \textit{any} explicit Runge--Kutta method. Such a function
would work by accepting the method parameters (in the form of a Butcher tableau)
as arguments. The top level function would then simply set the correct method
parameters before calling the sub-function.
If you choose to implement this method, then try experimenting with more
complicated methods (email me for examples).

%=================================================
\pagebreak

\section*{Details: \texttt{hw\_03\_studentName.m}}

\lstinputlisting{hw_03_studentName.m}

\subsection*{System Dynamics}

For this assignment you will need to simulate two systems:
the driven-damped pendulum and a passive double pendulum.
You will find functions that implement the system dynamics for both of these systems
in the code library for the course:\\
\texttt{ME149/codeLibrary/modelSystems/systemName/systemNameDynamics.m}\\
You can add the entire code library to your matlab path by running:\\
\texttt{ME149/codeLibrary/addLibraryToPath.m}\\
\\
You are permitted (and encouraged) to use any file
within the code library for your assignment.
You do not need to upload these files to Trunk.

\subsection*{Double Pendulum Dynamics}

The double pendulum dynamics have a different set of arguments
than are required by the standard simulation functions.
Use a function handle to pass through the correct arguments.

\vspace{-0.6em} \begin{itemize}  \setlength\itemsep{0em}
  \item \textbf{time} --- doublePendulumDynamics does not use time, you can ommit it.
  \item \textbf{control} --- doublePendulumDynamics accepts control torques, as well as the state.
    To make the system passive you will need to set both control torques to zero.
  \item \textbf{parameterss} --- doublePendulumDynamics requires that you pass system parameters,
    which can all be set to one. Use \texttt{>> help doublePendulumDynamics} to find the parameter names.
\end{itemize}

\textbf{Note:} If you've taken a dynamics course,
then I suggest that you take a look at the
script for deriving the dynamics function using the Matlab symbolic toolbox.
The techniques used there can be applied to any continuous dynamical system,
with a few small modifications, provided that it lives in two-dimensions.

\subsection*{Solution with \texttt{ode45()}}

You will need to compute the error associated with each
simulation method that you implement in \texttt{runSimulation.m}.
We will use Matlab's \texttt{ode45} function to approximation the true solution.
To do this, use \texttt{ode45} to run a simulation for each system, and
specify \texttt{'RelTol' = 1e-10} and \texttt{'AbsTol' = 1e-10}
using the \texttt{options} argument and \texttt{odeset}.

\subsection*{Simulation Parameters}

The driven-damped pendulum simulation should
start at $t = 0$ and finish at $t = 20$ using $150$ grid points.
Its initial angle should be $1.2$ radians and the
initial angular rate should be $0.8$ radians per second.

The passive double pendulum simulation should have a
duration of $10$ seconds and use $100$ grid points.
The initial angles for first link should be $0.6$
and the second link should be $0.9$.
Both links are initially stationary.
Link lengths, masses, and gravity should all be set to one.

\subsection*{Plots}

Create one figure for the driven-damped pendulum results and a
second figure for the passive double pendulum results.
Both figures should have the same formatting and outline.

\par
The top row of subplots should show the state while the bottom row shows the error.
Each subplot should have four curves: one for the solution (via \texttt{ode45}),
and the remaining three for each method that you implemented in
\texttt{runSimulation.m}: first-, second-, and forth-order
Be sure to include axis labels and a legend for each subplot.

\par The error plots should show the difference between your simulations and \texttt{ode45}.
You should plot the absolute value of the error, and the y-axis should be on a log scale.
Since the tolerance for \texttt{ode45} is set to $1e-10$, it makes not sense to have
any error values smaller than that: please clamp the error curves to have a minimum value of $1e-10$.

\par Following this outline,
the driven-damped pendulum figure will have a two-by-two
array of subplots, while the passive double pendulum figure will have
two rows and four columns of subplots.
Be sure to make your figures full-screen before saving them to pdf.

%=================================================
\end{document}
