% This file contains the header data for all assignment files
\documentclass[onecolumn, 11pt]{article}

\usepackage{bbold}   %Get fancy double struck math notation for sets
\usepackage{cite}
\usepackage{cleveref}
\usepackage{color}
\usepackage{courier}   %Have code written out nicely
\usepackage{float}
\usepackage[top=1in, bottom=1in, left=1in, right=1in]{geometry}
\usepackage{graphicx} % handles graphics and figures
\usepackage{hyperref}
\usepackage{listings}
\usepackage{mathtools,bm}
\usepackage{multicol}

\catcode`\^^M=10      %  Makes blank lines meaningless, force use of \par

\definecolor{magenta}{rgb}{0.8,0.0,1.0}
\definecolor{darkGreen}{rgb}{0.0,0.4,0.0}
\definecolor{blue}{rgb}{0.0, 0.0, 0.9}
\definecolor{purple}{rgb}{0.7, 0.0, 0.7}
\definecolor{darkGreen}{rgb}{0.0,0.4,0.0}

\newcommand{\quotes}[1]{``#1''}
\newcommand{\todo}[1]{{\color{magenta}\par {[{\bf ToDo: } {\em #1}} ] \\    }}

\newcommand{\norm}[1]{\left\lVert#1\right\rVert}

%%%%%%%%%%%%%%%%%%%%%%%%%%%%%%%%%%%%%%%%%%%%%%%%%%%%%%%%%%%%%%%%%%%%%%%%%%%%%%%
% NOTE:
%
% The following block of commands is used to format Matlab code blocks for the
% listings package. This block of code is based on two examples:
%  -->   https://gist.github.com/eyliu/120689
%  -->   http://links.tedpavlic.com/ascii/homework_new_tex.ascii
%
% Import file block using:
%   \lstinputlisting{fileName.m}
%
\lstloadlanguages{Matlab}
%
\lstset{language=Matlab,                        % Use MATLAB
        frame=single,                           % Single frame around code
        basicstyle=\small\ttfamily,             % Use small true type font
        keywordstyle=[1]\color{blue}\bfseries,  % MATLAB functions bold and blue
        keywordstyle=[2]\color{purple},         % MATLAB function arguments purple
        keywordstyle=[3]\color{blue}\underbar,  % User functions underlined and blue
        identifierstyle=,                       % Nothing special about identifiers
                                                % Comments small dark green courier
        commentstyle=\usefont{T1}{pcr}{m}{sl}\color{darkGreen}\small,
        stringstyle=\color{darkGreen},            % Strings are purple
        showstringspaces=false,                 % Don't put marks in string spaces
        tabsize=4,                              % 4 spaces per tab
        %
        %%% Put standard MATLAB functions not included in the default
        %%% language here
        morekeywords={xlim,ylim,var,alpha,factorial,poissrnd,normpdf,normcdf},
        %
        %%% Put MATLAB function parameters here
        morekeywords=[2]{on, off, interp},
        %
        %%% Put user defined functions here
        morekeywords=[3]{FindESS, homework_example},
        %
        morecomment=[l][\color{blue}]{...},     % Line continuation (...) like blue comment
        numbers=left,                           % Line numbers on left
        firstnumber=1,                          % Line numbers start with line 1
        numberstyle=\tiny\color{blue},          % Line numbers are blue
        stepnumber=5                            % Line numbers go in steps of 5
        }
%
%%%%%%%%%%%%%%%%%%%%%%%%%%%%%%%%%%%%%%%%%%%%%%%%%%%%%%%%%%%%%%%%%%%%%%%%%%%%%%%


%========================================================================
\title{Assignment 08:  Pendulum Swing-Up}
\date{Assigned:  March 16  ---  Due:  March 28 at 11:55pm}
\author{Optimal Control for Robotics}
%========================================================================
\begin{document}
\maketitle
%=================================================

\section*{Introduction}

In this assignment you will compute the optimal swing-up trajectory for a simple
pendulum using multiple shooting. We will keep things simple:
use Euler's method on a uniform time grid,
with one simulation step per multiple shooting segment.

\section*{Deliverables}

Implement the function \texttt{simplePendulumOptimBvp} using the template provided.

\section*{Write-Up:}

There is no separate write-up for this assignment. Instead,
please include the total time you spent working on this assignment in
the comments near the top of your implementation of\\
\texttt{simplePendulumOptimBvp}.
Be sure to clearly organize and document your code.

\section*{Comments}

In this assignment I provide a simple script to run your optimization,
along with an encrypted version of the solution.
You should run this script with a variety of different parameters to learn
more about how trajectory optimization behaves with different inputs.
Once you've done that, then start working on implementing your code.

\par If you run out of time and your assignment does not work correctly,
then add a notes section to the comments in the top of \texttt{simplePendulumOptimBvp}.
You should clearly state that the function does not work, why you think it does
not work, and your best guess about how to go about debugging it if you had time.

\par The file \texttt{simplePendulumOptimBvpSoln.p} provides you with a correct
implementation of multiple shooting, against which you can compare your code.
This implementation is not guarenteed to obtain the \quotes{true} solution to
the trajectory optimization problem. If FMINCON converges, then the resulting
trajectory is an approximation of a locally optimal trajectory.

\par One final note: \texttt{simplePendulumOptimBvpSoln.p} uses a very simple
initialization routine. If you are clever you can probably beat it: either by
finding the same solution more quickly or by avoiding a local minima.
For example, the optimal solution might require three swings,
but \texttt{simplePendulumOptimBvpSoln.p} only manages to find a solution with
one swing (which has a higher objective function value).

%=================================================
\end{document}
