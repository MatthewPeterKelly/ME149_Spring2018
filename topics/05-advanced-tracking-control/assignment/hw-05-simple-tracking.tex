% This file contains the header data for all assignment files
\documentclass[onecolumn, 11pt]{article}

\usepackage{bbold}   %Get fancy double struck math notation for sets
\usepackage{cite}
\usepackage{cleveref}
\usepackage{color}
\usepackage{courier}   %Have code written out nicely
\usepackage{float}
\usepackage[top=1in, bottom=1in, left=1in, right=1in]{geometry}
\usepackage{graphicx} % handles graphics and figures
\usepackage{hyperref}
\usepackage{listings}
\usepackage{mathtools,bm}
\usepackage{multicol}

\catcode`\^^M=10      %  Makes blank lines meaningless, force use of \par

\definecolor{magenta}{rgb}{0.8,0.0,1.0}
\definecolor{darkGreen}{rgb}{0.0,0.4,0.0}
\definecolor{blue}{rgb}{0.0, 0.0, 0.9}
\definecolor{purple}{rgb}{0.7, 0.0, 0.7}
\definecolor{darkGreen}{rgb}{0.0,0.4,0.0}

\newcommand{\quotes}[1]{``#1''}
\newcommand{\todo}[1]{{\color{magenta}\par {[{\bf ToDo: } {\em #1}} ] \\    }}

\newcommand{\norm}[1]{\left\lVert#1\right\rVert}

%%%%%%%%%%%%%%%%%%%%%%%%%%%%%%%%%%%%%%%%%%%%%%%%%%%%%%%%%%%%%%%%%%%%%%%%%%%%%%%
% NOTE:
%
% The following block of commands is used to format Matlab code blocks for the
% listings package. This block of code is based on two examples:
%  -->   https://gist.github.com/eyliu/120689
%  -->   http://links.tedpavlic.com/ascii/homework_new_tex.ascii
%
% Import file block using:
%   \lstinputlisting{fileName.m}
%
\lstloadlanguages{Matlab}
%
\lstset{language=Matlab,                        % Use MATLAB
        frame=single,                           % Single frame around code
        basicstyle=\small\ttfamily,             % Use small true type font
        keywordstyle=[1]\color{blue}\bfseries,  % MATLAB functions bold and blue
        keywordstyle=[2]\color{purple},         % MATLAB function arguments purple
        keywordstyle=[3]\color{blue}\underbar,  % User functions underlined and blue
        identifierstyle=,                       % Nothing special about identifiers
                                                % Comments small dark green courier
        commentstyle=\usefont{T1}{pcr}{m}{sl}\color{darkGreen}\small,
        stringstyle=\color{darkGreen},            % Strings are purple
        showstringspaces=false,                 % Don't put marks in string spaces
        tabsize=4,                              % 4 spaces per tab
        %
        %%% Put standard MATLAB functions not included in the default
        %%% language here
        morekeywords={xlim,ylim,var,alpha,factorial,poissrnd,normpdf,normcdf},
        %
        %%% Put MATLAB function parameters here
        morekeywords=[2]{on, off, interp},
        %
        %%% Put user defined functions here
        morekeywords=[3]{FindESS, homework_example},
        %
        morecomment=[l][\color{blue}]{...},     % Line continuation (...) like blue comment
        numbers=left,                           % Line numbers on left
        firstnumber=1,                          % Line numbers start with line 1
        numberstyle=\tiny\color{blue},          % Line numbers are blue
        stepnumber=5                            % Line numbers go in steps of 5
        }
%
%%%%%%%%%%%%%%%%%%%%%%%%%%%%%%%%%%%%%%%%%%%%%%%%%%%%%%%%%%%%%%%%%%%%%%%%%%%%%%%


%========================================================================
\title{Assignment 5:  Quadrotor Control via LQR}
\date{Assigned:  Feb 13  ---  Due:  Feb 27}
\author{Tufts ME 149:  Optimal Control For Robotics}
%========================================================================
\begin{document}
\maketitle
%=================================================

\section*{Introduction}

In this assignment you will design two controllers for a
planar model of a quadrotor helicopter.
The first controller will be designed using infinite-horizon LQR and
will be used to regular the state of the quadrotor in a stationary hover.
The second controller will be designed using finite-horizon (trajectory) LQR
and will stabilize an arbitrary reference trajectory that starts and ends in
a static hover.

\par
Unlike previous assignments, you will not need to submit any plots.
Instead, for each part of the assignment you will submit a Matlab function
that returns a function handle for a feedback controller.
Your controller will be evaluated by a set of automated unit tests.
These tests will run simulations of the closed loop system with a variety of
different perturbations, disturbances, and other error sources.

\par
You should write your own test scripts to ensure that your controllers work.
There are no direct requirements on these tests, although you should submit them
along with your other code. A few things that you could test,

\section*{Planar Quadrotor Model}

This assignment is centered around controlling a planar quadrotor model.
All of the dynamics functions that you will need are provided in the ME-149
code library:\\
\texttt{ME149/codeLibrary/modelSystems/planarQuadrotor/}.

\par
There are several parameters for the quadrotor. Your controller design function
should work for any valid set of parameters, but you may choose the following
set of nominal values for your tests:
\texttt{param.m = 0.4}, \texttt{param.w = 0.4}, \texttt{param.g = 10}

\par
In addition to the dynamics functions, the code library also provides a variety
of useful diagnostics tools for the quadrotor. For example, you can visualize the
output of a simulation using the
\texttt{planarQuadrotorPlot()} to generate a plot of the state and control versus time,
or \texttt{planarQuadrotorAnimate()} to generate an interactive animation.

\par
For the purposes of testing, I have included a special version of the dynamics
function:\\
\texttt{planarQuadrotorRealDyn()} that implements a more realistic
version of the dynamics. In particular, it adds a random disturbance force that
is a rough proxy for wind. By default it will always use the same disturbance,
but you can pass a seed to the random number generator to test additional disturbances.
You can also adjust the magnitude of the disturbances (see the help file for details).


\section*{Part One: Hover Controller}

For the first part of this assigment you will need to implement the file
\texttt{getHoverController.m}, included below. The output of this function is
a feedback controller (as a function handle) that can be passed into a simulation.

Also write a test for the controller (\texttt{TEST\_getHoverController.m}).
This function (or script) should generate one or more simulations that test
your controller and demonstrate that it is reliable.

\pagebreak
\lstinputlisting{getHoverController.m}

\section*{Part Two: Trajectory Tracking}

The second part of this assignment is similar to the first, except that now you
will design a trajectory tracking controller instead of a hovering controller.
As with the first part, you do not need to submit plots - instead you will
implement two matlab functions: \\
\texttt{getTrackingController.m} \\
\texttt{trajectoryLqr.m} \\

\par
The first function will design a trajectory tracking controller,
and the second function computes the time-varying LQR gains along a reference trajectory.
The details for implementation are included in the files.
Your controller will be evaluated by running it through a series of automated
test simulations.

\par
You should thoroughly test you controller before submitting to make sure that it works.
Submit the test script (\texttt{TEST\_getTrackingController.m}) to Trunk along with the rest of your code.
The only requirement for this test script is that it
should demonstrate that you controller works by running one or more simulations
and then generating some meaningful output.

\par
The assignment includes an single reference trajectory that guides the quadrotor
through a back-flip while moving horizontally between two static hover poses.
The function \texttt{importReferenceTrajectory.m} can be used to import this data and convert
it into a useful format.

\section*{Write-Up}

In addition to the Matlab files listed above, please also upload a short write-up
for this assignment: \\
\texttt{hw\_05\_studentName\_writeup.txt}
\vspace{-0.0em} \begin{itemize}  \setlength\itemsep{0em} \setlength\itemindent{18pt}
  \item Header: full name, date, assignment name and number
  \item List any other students that you worked with.
  \item How long did this assignment take you?
  \item Briefly describe your approach to testing your controllers.
\end{itemize}

%=================================================
\end{document}
